\documentclass{article}

%\usepackage[spanish]{babel}

\usepackage[utf8]{inputenc}

\usepackage{graphicx}

\title{Ecuaciones en Diferencias}

\author{Dalia Itzel Espinosa Tolentino}

\date{2 de octubre de 2017}

\begin{document}

\maketitle


\section{Ecuaciones de Primer Orden}

\subsection{Ecuaciones Liniales}

Una ecuación en diferencias de primer grado tiene la forma $x_{n+1}=ax_n$ donde $a$ es una constante.

La formula para resolver ecuaciones lineales es 
\begin{equation}
  \label{eq:eq2}
x_n=a^nx_0
\end{equation}

Por ejemplo si iniciamos una inversión con 1000 pesos con un interés mensaul del 1\%, obtenemos lo siguiente:

\begin{center}
  \includegraphics[width=8cm]{inversion.png}
\end{center}


\subsection{Ecuaciones Afines}

Una ecuación afin en diferencias de primer grado tiene la forma $x_{n+1}=ax_n+b$ donde $a$ es una constante.

La formula para resolver ecuaciones afines es

\begin{equation}
  \label{eq:eq1}
  x_n=a^n(x_n-\alpha)+\alpha
\end{equation}

donde $\alpha=\frac{b}{1-a}$.

Para deducir esta fórmula usamos que $$\sum_{i=0}^{n-1}a^i=\frac{a^n-1}{a-1}$$

\section{Ecuaciones de Segundo Orden}
El metodo para resolver estas ecuaciones esta 
%El método para resolver estas ecuaciones está inspirado en la formula \ref{eq:eq2}.
 
\end{document}


% Una ecuación en diferencias de primer grado tiene la forma $x_{n+1}=ax_n$ donde $a$ es una constante.
%  ecuación afin en diferencias de primer grado tiene la forma $x_{n+1}=ax_n+b$ donde $a$ es una constante.